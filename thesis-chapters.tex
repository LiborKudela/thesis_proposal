\chapter{Úvod}
Vývoj centrálního zásobování teplem prošel od dob svého vzniku několika fázemi.
Z hlediska dnešní doby je nejvýznamější býměna starých parovodů za
prefabrikované horkovody, čímž dochází ke snížení přestupu tepla z rozvodů do
jejich okolí a snížemí nákladů při stavbě nových částí sitě. Z hlediska
několika budoucích dekád je žádoucí transformace na obnovitelné, plně
udržitelné a efektivní zásobování teplem. To přináší nové technologické výzvy
související zejeména s integrací obnovitelných zdrpjů, využitím odpadního tepla
a celkově snižováním emisí oxidu uhličitého. Je totiž obecně přijímaným faktem,
že pro smysluplnou implementaci obnovitelných zdrojů tepla a odpadní, jež lze
obecně lze obecně považovat za nízkoteplotní zdroje, je nutné dále snižovat
teplotu v teplárenských rozvodech. Snižovaní této teploty má za následek další
redukci ztrát při distribuci tepla, ale komplikuje efektivní předávání tepla na
straně spotřebitelů a zvyšuje provozní náklady čerpadel. Proto je nutné do sítě
implementovat další prvky, jež umožní efektivní zvýšení teploty blízko
samotných spotřebitelů. Jako další problém lze zmínit značnou nestálost
obnovitelných zdrojů tepla. Se zvyšující se komlexností sítí bude nutné zlepšit
možnosti optimalizace jejich provozu. Toho lze dosáhnout například prediktivním
řízením využívající matematické modely a vhodnou koncepcí akumulace tepla
společně s předpovědí počasí a predikcí množství potřebného tepla. Systémy, jež
se budou schopny vypořádat s těmito a dalšími výzvami lze obecně nazvat čtvrtou
generací teplárenských sítí \cite{Lund2014}.

Množství hmoty, které ovlivňuje fyzikální dynamiku rozsáhlých energetických 
systémů již samo o sobě naznačuje komplexnost potencionálního matematického 
modelu (digitálního dvojčete či prototypu). Užitečnost takových modelů roste 
společně s jejich výpočtovou rychlostí a také s rychlostí s jakou je možné tyto 
modely sestavovat. Deklarativní programování může výrazně zjednodušit proces 
kompletace z jednotlivých komponent. V numerickém řešení pak modely obvykle 
využívají matice, jež se v nich objevují zpravidla kvůli tzv. linearizaci 
diferenciálních rovnic (tedy rovnic řídících fyzikální chování). Výpočtová 
rychlost je pak úměrná velikosti a hustotě těchto matic. V dnešní době se 
můžeme často setkat se snahou o zmenšováním matic pomocí redukce směrem k tzv.
jedno‑dimenzionálním modelům. Dalším výrazný vliv na výpočetní rychlost mají
programovací jazyk použitý pro implementaci (strojový kód zpravidla poběží 
rychleji než kód interpretovaný) a také míra a provedení paralelizace
algoritmů.

Počítačové učení je moderní vědecká disciplína, která umožňuje vytvářet modely 
pomocí dat a to na základě optimalizace parametrizovaných programů. V této 
oblasti existuje celá řada numerických struktur umožňující až univerzální 
aproximaci (např. neuronové sítě).

Cílem této disertační práce je seskupení/vytvoření nástrojů (softwarových 
balíčků) pro modelování komponent tepelného zásobování s ohledem na 
optimalizaci numerické efektivnosti simulací s využitím deklarativního 
programování a počítačového učení. Model tepelného zásobování zde poslouží jako 
studijní příklad jejich aplikace.

\chapter{Abstrakt}
\label{abstrakt}
Text abstraktu.

\chapter{Současné shrnutí stavu poznání} 
\label{struktura}
V této kapitole je provedena rešerše problematiky modelování CZT a popis 
vybraných matematických rovnic aplikovatelných pro daný kontext. Dále jsou 
popsány použitelné výpočetní přístupy spolu s aspekty, ovlivňující numerickou 
efektivnost. 

\section{Stručná historie rozvoje teplárenství}
Teplárenské potrubí slouží k dopravě tepla od jeho výrobce k jeho spotřebiteli 
pomocí teplonosného média. V minulosti se jako první teplonosné médium 
používala pára o teplotě i přes 240 °C. Tyto rozvody generovaly značné tepelné 
ztráty a kvůli vysokému tlaku (naakumulované tlakové energii) i vážné 
explozivní nehody. Kvůli kondenzaci páry ve vratném potrubí docházelo k jeho 
korozi. Pára se jako hlavní teplonosné médium stále využívá například v Paříži 
či Manhattanu. Nástupce tohoto provedení využíval tlakovou horkou vodu většinou 
přes 100 °C. Kvůli nestlačitelnosti vody se výrazně omezily explozivní nehody. 
Motivací bylo snížení tepelných ztrát a možnost lepší implementace kombinované 
výroby tepla a elektrické energie v městských oblastech. Poté přišel trend 
snižování teploty v rozvodech pod 100 °C. Stále byla využívána tlaková horká 
voda. Začalo se značně využívat prefabrikované a před-izolované potrubí 
snižující množství lidské práce při výstavbě a obnově rozvodů. Začala se 
využívat lokální paliva, odpady a na několika místech i sluneční či geotermální 
energie. V následujících dvou až třech dekádách bude trend snižování teploty v 
rozvodech pokračovat ruku v ruce se snižující se náročností budov na prostorové 
vytápění. Bude existovat snaha o výraznější implementaci obnovitelných a 
odpadních zdrojů energie a k využívání synergií. Pro snížení následků fluktuací 
zdrojů, budou využívány akumulační systémy společně s prediktivním řízením 
využívající předpovědi meteorologických podmínek a matematické modely 
jednotlivých součástí systému se zaměřením na jejich dynamiku \cite{Lund2014}.

\section{Fyzika teplárenských rozvodů/potrubí}
Moderní teplárenské systémy s teplonosným médiem v kapalném skupenství. Fluidní 
dynamika je prostorovým fenoménem, ale lze zavádět zjednodušující předpoklady 
odpovídající kontextu aplikace. Při uvažování zjednodušené fyziky proudění 
tekutiny v potrubí (tj. na nestlačitelné a jedno-dimenzionální proudění) lze 
problém zjednodušit na součinnost třech fyzikálních jevů:

\begin{itemize}
\item Advekce veličin jež jsou unášeny proudem podél potrubí 
\item Difůze veličin v tekutině podél potrubí
\item Příspěvky do veličin vnitřními a/či externími zdroji
\end{itemize}

\begin{figure}[h] \capstart
\includegraphics[width=\textwidth]{figures/heat_front}
\caption{Proces šíření teplotní vlny v potrubí \cite{Abraham2009}}
\end{figure}

Parciální diferenciální rovnice (PDR) pro celkovou dynamiku je pak v tomto 
kontextu následující:
\begin{equation} 
\frac{dy}{dt} = -u \frac{dy}{dx} + D\frac{d^{2}y}{dx^2} + {S_y}  
\end{equation}
kde \(y\) je unášená veličina, \(t\) je čas, \(x\) je podélná poloha, \(u\) je
rychlost tekutiny, \(D\) je celkový koeficient difúze a \(Sy\) je celkový
zdroj unášené veličiny. Z hlediska tepelného chování může být touto veličinou 
měrná entalpie, ale lze pracovat i s teplotou.

Advekce je jev, při kterém se transportují veličiny (jsou unášeny) ve směru
objemového pohybu. Tento jev ovlivňuje tepelnou dynamiku nejvíce, neboť z velké
části určuje čas transportu (zpoždění) teplotní fronty.

Intenzita axiální difúze (velikost koeficientu D) je složena jednak z difúze
kodkukcí tepla v tekutině a dále z efektu tzv. turbulentního míchání v axiálním
směru. Difúze způsobená kondukcí je u teplárenských rozvodů zanedbatelná [x].
Za to turbulentní difúze roste dle přibližně lineárně s Reynoldsovým číslem
[x].  Axiální difúze způsobuje rozpínání (vyhlazování) teplotní fronty, což
znamená, že náhlé změny na vstupu do potrubí se projevují pozvolnými změnami na
jeho konci. V případě uvažování horizontálně uloženého potrubí lze tímto
způsobem zahrnout i vliv gravitace. Nemá tak vliv na zpoždění teplotní fronty,
ale výrazně ovlivňuje její tvar.

Zdrojový termín Sy(x) zahrnuje především výměnu tepla mezi tekutinou a vnitřním
povrchem přilehlé stěny ve kterém tekutina proudí (což v důsledku určuje vliv
akumulace tepla na vývoj teplotní fronty). Velikost hustoty tepelného toku je
závislá na rozdílu teploty vody a povrchu, vlastnostech tekutiny, velikosti a
drsnosti povrchu a na rychlosti proudění. Výměna tepla ovlivňuje jak zpoždění
teplotní fronty tak její tvar. Intenzitu výměny tepla lze vyjádřit zcela běžnou
konvekcí,

\begin{equation} 
Q = \alpha S \Delta T 
\end{equation}
kde \(Q\) je tepelný tok ze stěny do tekutiny,\(\alpha\) je součinitel přestupu
tepla, \(S\) je velikost plochy kde je styčného povrchu a \(\Delta T\) je 
rozdíl teplot povrchu a tekutiny. Koeficient přenosu tepla  je možné určovat z
tzv. Nusseltova čísla, které vyjadřuje poměr mezi konvektivním a koduktivním 
přestupem tepla. Známe-li tedy množství tepla, jež by procházelo mezní vrstvou 
v případě čisté kondukce tepla je možné určit množství tepla v případě 
konvektivního přestupu tepla:

\begin{equation} 
Nu = \frac{\alpha L}{\lambda} 
\end{equation}
kde je \(Nu\) Nusseltovo číslo, \(L\) je charakteristický rozměr (pro kruhové 
potrubí je jím vnitřní průměr) a \(\lambda\) je tepelná vodivost tekutiny.

\begin{figure}[h] \centering \capstart
\includegraphics[scale=0.3]{figures/nusselt}
\caption{Závislost Nusseltova čísla na Reynoldsově čísle \cite{Abraham2009}}
\end{figure}

Hodnotu Nusseltova čísla je možné určit na základě vztahů jak pro rovnoměrné 
rozložení teploty na stěně (označováno jako UWT), tak pro rovnoměrné rozložení
tepelného toku (označované jako UHF). Dle \cite{Abraham2009} je pro všecny
režimi proudění vhodné používat následující model Nusseltova čísla:
\begin{equation}
Nu = 
\begin{dcases}
  c_0 & Re\leq 2300 \\
  c_1\left(\frac{Re}{1000}\right)^4
  -c_2\left(\frac{Re}{1000}\right)^3
  +c_3\left(\frac{Re}{1000}\right)^2
  -c_4\left(\frac{Re}{1000}\right)
  +c_5 & 2300 < Re\le 3100 \\
  \frac{\frac{f}{8}(Re-1000)Pr}
  {1+12.7{\frac{f}{8}}^{(1/2)}(Pr^{(2/3)}-1)} & 3100 < Re
\end{dcases} 
\end{equation}
kde \(Pr\) je Prandtlovo číslo (vlastnost tekutiny) a \(f\) je Darcy-Weisbachův 
koeficient tření který je popsán v podkapitole o tlakových ztrátách (skrze něj 
se projevuje vliv drsnosti). Jednotlivé hodnoty parametrů tohoto modelu jsou:

\begin{table}[H]
  \vskip6pt
  \caption{Parametry modelu Nusseltova čísla}
    \vskip6pt
  \centering
  \begin{tabular}{ccc}
    \toprule
    Konstanta & UWT & UHF \\ [0.5ex]
    \hline
    \(c_0\) & 3,66 & 4,36 \\ 
    \(c_1\) & 3,52 & 2,2407 \\
    \(c_2\) & 45,148 & 29,499 \\
    \(c_3\) & 212,13 & 142,32 \\
    \(c_4\) & 427,45 & 292,51 \\
    \(c_5\) & 316,08 & 219,88 \\
    \bottomrule \\[0.1mm]
  \end{tabular}
  \label{tab:ExampleTable2}
\end{table}

V případě, že je teplota vnitřního povrchu potrubí
známa je možné na základě rovnic (2-5) určit intenzitu výměny tepla v
konkrétním okamžiku. Tato teplota je však ovlivněna tepelnou hmotou jež
obklopuje fluidní region (např. ocelová trubka, izolace, zemina apod.) a tudíž
je je nutno korektně matematicky zachytit (viz podkapitola 5.2.3).
kde jednotlivé konstany jsou:
\chapter{Jednotlivé kapitoly práce}
\label{kapitoly}

\section{Obsah}
\label{obsah}

\section{Úvod}
\label{uvod}

\textbf{Příklad}:
\begin{verbatim}
   @Article{Cech:2020:Citace,
	   author               = "Čech, Jan",
	   key                  = "Cech",
	   ... 
\end{verbatim}


